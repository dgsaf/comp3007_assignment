\documentclass[draft]{article}

% - style template
\usepackage{base}

% - title, author, etc.
\title{COMP3007 - Assignment}
\author{Tom Ross}
\date{\today}

% - headers
\pagestyle{fancy}
\fancyhf{}
\rhead{\theauthor}
\chead{}
\lhead{\thetitle}
\rfoot{\thepage}
\cfoot{}
\lfoot{}

% - document
\begin{document}

The entire code repository can be found at
\url{https://github.com/dgsaf/comp3007_assignment}.

\tableofcontents

\listoffigures

\listoftables

\clearpage

\section{Introduction}
\label{sec:introduction}

\section{Implementation}
\label{sec:implementation}

\subsection{Problem Analysis}
\label{sec:problem-analysis}

To detect the characters (digits and arrows), we utilise the technique of Maximally
Stable Extremal Regions (MSER).
We suggest that this technique is suitable for our purpose, as the characters
are uniformly white and contrast well against the surrounding uniformly black
area of the signs.
Hence, these characters should form regions which are very stable under
thresholding, when compared with their local context.
Furthermore, this technique should be fairly robust to the presence of shadows
and other variations in illumination.

\subsection{Technique}
\label{sec:technique}

\subsubsection*{Image}

We define an image $I$, with $n$ channels and of width $w$ and height $h$, to be
a mapping
\begin{equation}
  \label{eq:def-image}
  I
  :
  D \to S
  :
  \lr{x, y}
  \mapsto
  I\lr{x, y}
  =
  \lr[\big]
  {
    I_{1}\lr{x, y}
    ,
    \dotsc
    ,
    I_{n}\lr{x, y}
  }
  ,
\end{equation}
where the domain $D \subset \natural^{2}$ is of the form
$D = \lrset{0, \dotsc, w - 1} \times \lrset{0, \dotsc, h - 1}$, and where the
codomain $S$ is in general of the form $S \subset \real^{n}$.

\subsubsection*{Thresholded Image}

Suppose $I : D \to S \subset \real$ is a single-channel image.
For each $t \in S$, we define the Boolean-valued image $I_{t}$ to be of the
form
\begin{equation}
  \label{eq:def-image-threshold}
  I_{t}
  :
  D \to \mathbb{B}
  :
  \lr{x, y}
  \mapsto
  \begin{cases}
    0
    &\qq{if}
    I\lr{x, y} \leq t
    \\
    1
    &\qq{if}
    I\lr{x, y} > t
  \end{cases}
  ,
\end{equation}
and we say that $I_{t}$ is a thresholded image.

\subsubsection*{Connectedness}

Suppose $D$ is the domain of an image.
An adjacency relation $A$ on $D$ is a Boolean-valued mapping
\begin{equation}
  \label{eq:def-adjacency}
  A
  :
  D \times D \to \mathbb{B}
  :
  \lr{p, q}
  \mapsto
  A\lr{p, q}
  ,
\end{equation}
which indicates if the two points of the domain are considered to be adjacent.
For any $p \in D$, we may define the neighbourhood $N\lr{p}$ of $p$ to be the
set of all points which are adjacent to it; that is,
\begin{equation}
  \label{eq:def-neighbourhood}
  N\lr{p}
  =
  \lrset
  {
    q \in D
    \mid
    A\lr{p, q}
  }
  .
\end{equation}
For any $p, q \in D$, we say that $p$ and $q$ are connected if there exists a
finite sequence $\lr{\rho_{k}}_{1 \leq k \leq n}$ in $D$ such that
\begin{equation}
  \label{eq:def-connected}
  A\lr{p, \rho_{1}}
  \wedge
  A\lr{\rho_{1}, \rho_{2}}
  \wedge
  \dotsb
  \wedge
  A\lr{\rho_{n-1}, \rho_{n}}
  \wedge
  A\lr{\rho_{n}, q}
  =
  1
  .
\end{equation}
In the case where the adjacency relation $A$ is symmetric, then connectedness
defines an equivalency relation; whence we may write, for any connected
$p, q \in D$ that $p \sim q$.

\subsubsection*{Region}

We define image regions, for the case of a single channel image, in the theme of
the MSER approach.
Suppose $I : D \to S \subset \real$ is a single-channel image, and for all
$t \in S$, let $I_{t}$ be its thresholded image.
Suppose $A : D \times D \to \mathbb{B}$ is the (symmetric) adjacency relation
associated with either the Von Neumann neighbourhood (4-connectivity) or the
Moore neighbourhood (8-connectivity).
For all $t \in S$, we define a new adjacency relation
$A_{t} : D \times D \to \mathbb{B}$ by
\begin{equation}
  \label{eq:def-adjacency-threshold}
  A_{t}\lr{p, q}
  =
  A\lr{p, q}
  \wedge
  \lrsq[\big]
  {
    I_{t}\lr{p}
    \iff
    I_{t}\lr{q}
  }
  ,
\end{equation}
that is, two points $p, q \in D$ are considered adjacent by $A_{t}$ if they are
geometrically adjacent by $A$ and map to equivalent values in the thresholded
image $I_{t}$.

For all $t \in S$, we define an image region $R \subset D$ to be a subset of
the image domain that is connected by the adjacency relation $A_{t}$; that is,
for all $p, q \in R$, we have that $p \sim q$ by $A_{t}$.

\subsection{Task 1}
\label{sec:imp-task-1}

\subsection{Task 2}
\label{sec:imp-task-2}

\section{Validation Performance}
\label{sec:val-per}

\subsection{Task 1}
\label{sec:val-task-1}

\subsection{Task 2}
\label{sec:val-task-2}

\section{Conclusion}
\label{sec:conclusion}

\clearpage

\appendix

\section{Source Code}
\label{sec:source-code}

\end{document}
