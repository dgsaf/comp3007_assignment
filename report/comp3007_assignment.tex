\documentclass[draft]{article}

% - style template
\usepackage{base}

% - title, author, etc.
\title{COMP3007 - Assignment}
\author{Tom Ross}
\date{\today}

% - headers
\pagestyle{fancy}
\fancyhf{}
\rhead{\theauthor}
\chead{}
\lhead{\thetitle}
\rfoot{\thepage}
\cfoot{}
\lfoot{}

% - document
\begin{document}

The entire code repository can be found at
\url{https://github.com/dgsaf/comp3007_assignment}.

\tableofcontents

\listoffigures

\listoftables

\clearpage

\section{Introduction}
\label{sec:introduction}

\section{Implementation}
\label{sec:implementation}

\subsection{Problem Analysis}
\label{sec:problem-analysis}

To detect the characters (digits and arrows), we utilise the technique of Maximally
Stable Extremal Regions (MSER).
We suggest that this technique is suitable for our purpose, as the characters
are uniformly white and contrast well against the surrounding uniformly black
area of the signs.
Hence, these characters should form regions which are very stable under
thresholding, when compared with their local context.
Furthermore, this technique should be fairly robust to the presence of shadows
and other variations in illumination.

\subsection{Technique}
\label{sec:technique}

\subsubsection*{Image}

We define an image $I$, with $n$ channels and of width $w$ and height $h$, to be
a mapping
\begin{equation}
  \label{eq:def-image}
  I
  :
  D \to S
  :
  \lr{x, y}
  \mapsto
  I\lr{x, y}
  =
  \lr[\big]
  {
    I_{1}\lr{x, y}
    ,
    \dotsc
    ,
    I_{n}\lr{x, y}
  }
  ,
\end{equation}
where the domain $D \subset \natural^{2}$ is of the form
$D = \lrset{0, \dotsc, w - 1} \times \lrset{0, \dotsc, h - 1}$, and $S$ is the
codomain.
Constraints can be specified on $S$, but for our purposes it suffices to suppose
that either $S \subseteq \lrset{0, \dotsc, 255}^{n}$ or $S \subset \real^{n}$.

\subsubsection*{Thresholded Image}

Suppose $I : D \to S \subset \real$ is a single-channel image.
For each $t \in S$, we define the Boolean-valued image $I_{t}$ to be of the
form
\begin{equation}
  \label{eq:def-image-threshold}
  I_{t}
  :
  D \to \mathbb{B}
  :
  \lr{x, y}
  \mapsto
  \begin{cases}
    0
    &\qq{if}
    I\lr{x, y} \leq t
    \\
    1
    &\qq{if}
    I\lr{x, y} > t
  \end{cases}
  ,
\end{equation}
and we say that $I_{t}$ is a thresholded image.


\subsubsection*{Bounding Box}

Suppose $D$ is the domain of an image $I : D \to S$.
For all $R \subseteq D$, we define the bounding box $B\lr{R}$ of $R$ to be
\begin{equation}
  \label{eq:def-bounding-box}
  B\lr{R}
  =
  \lrsq{x_{R}, x_{R} + w_{R}}
  \times
  \lrsq{y_{R}, y_{R} + h_{R}}
  \qq{such that}
  R \subseteq B\lr{R}
\end{equation}
and where $w_{R}, h_{R} \in \natural$ are minimal.

\subsubsection*{Connectedness}

Suppose $D$ is the domain of an image.
An adjacency relation $A$ on $D$ is a Boolean-valued mapping
\begin{equation}
  \label{eq:def-adjacency}
  A
  :
  D \times D \to \mathbb{B}
  :
  \lr{p, q}
  \mapsto
  A\lr{p, q}
  ,
\end{equation}
which indicates if the two points of the domain are considered to be adjacent.
For any $p \in D$, we may define the neighbourhood $N\lr{p}$ of $p$ to be the
set of all points which are adjacent to it; that is,
\begin{equation}
  \label{eq:def-neighbourhood}
  N\lr{p}
  =
  \lrset
  {
    q \in D
    \mid
    A\lr{p, q}
  }
  .
\end{equation}
For any $p, q \in D$, we say that $p$ and $q$ are connected if there exists a
finite sequence $\lr{\rho_{k}}_{1 \leq k \leq n}$ in $D$ such that
\begin{equation}
  \label{eq:def-connected}
  A\lr{p, \rho_{1}}
  \wedge
  A\lr{\rho_{1}, \rho_{2}}
  \wedge
  \dotsb
  \wedge
  A\lr{\rho_{n-1}, \rho_{n}}
  \wedge
  A\lr{\rho_{n}, q}
  =
  1
  .
\end{equation}
In the case where the adjacency relation $A$ is symmetric, then connectedness
defines an equivalency relation; whence we may write, for any connected
$p, q \in D$ that $p \sim q$.

Note that we are primarily concerned with the adjacency relations associated
with the Von Neumann neighbourhood (4-connectivity)
\begin{equation}
  \label{eq:def-neighbourhood-4}
  N_{4}\lr{p}
  =
  \lrset
  {
    p + n \in D
    \mid
    n
    \in
    \lrset
    {
      \lr{0, 1}
      ,
      \lr{1, 0}
      ,
      \lr{0, -1}
      ,
      \lr{-1, 0}
    }
  }
  ,
\end{equation}
and the Moore neighbourhood (8-connectivity)
\begin{equation}
  \label{eq:def-neighbourhood-8}
  N_{8}\lr{p}
  =
  \lrset
  {
    p + n \in D
    \mid
    n
    \in
    \lrset{-1, 0, 1} \times \lrset{-1, 0, 1}
    \setminus \lr{0, 0}
  }
  .
\end{equation}

\subsubsection*{Region}

Suppose $D$ is the domain of an image $I : D \to S$ and let
$A : D \times D \to \mathbb{B}$ be a symmetric adjacency relation on $D$.
We say that $R \subseteq D$ is a region if every element of $R$ is connected to
every other element of $R$; that is,
\begin{equation}
  \label{eq:def-region}
  p, q \in R
  \implies
  p \sim q
  .
\end{equation}
We define the (inner) boundary $\partial R$ of a region $R$ to be subset of
points of $R$ which are also connected to at least one point not in $R$; that
is,
\begin{equation}
  \label{eq:def-region-boundary-inner}
  \partial R
  =
  \lrset
  {
    p \in R
    \mid
    \exists q \in D \setminus R
    :
    A\lr{p, q}
  }
  .
\end{equation}
We define the outer boundary $\Delta R$ of a region $R$ to be the set of points
of $D$ which do not belong to $R$ but are adjacent to a point of $R$; that is,
\begin{equation}
  \label{eq:def-region-boundary-outer}
  \Delta R
  =
  \lrset
  {
    p \in D \setminus R
    \mid
    \exists q \in R
    :
    A\lr{p, q}
  }
  .
\end{equation}

\subsubsection*{Maximally Stable Extremal Region (MSER)}

Suppose $I : D \to S \subset \real$ is a single-channel image, and suppose
$A : D \times D \to \mathbb{B}$ is the (symmetric) adjacency relation associated
with either the Von Neumann neighbourhood or the Moore neighbourhood.
Suppose $R \subseteq D$ is a region.
We say that $R$ is a minimal region if for all $p \in R$ and $q \in \Delta R$ we
have $I\lr{p} < I\lr{q}$; which is equivalently written as the requirement that
\begin{equation}
  \label{eq:def-region-minimal}
  \max_{p \in R} I\lr{p}
  <
  \min_{q \in \Delta R} I\lr{q}
  .
\end{equation}
Similarly, we say that $R$ is a maximal region if for all $p \in R$ and
$q \in \Delta R$ we have $I\lr{p} > I\lr{q}$; which is equivalently written as
the requirement that
\begin{equation}
  \label{eq:def-region-minimal}
  \min_{p \in R} I\lr{p}
  >
  \max_{q \in \Delta R} I\lr{q}
  .
\end{equation}
We say that $R$ is an extremal region if it is either a minimal or maximal
region.

The formulation of extremal regions in terms of the minimal and maximal
intensity values of the image permits the usage of thresholding.
Suppose that $R$ is an extremal region, and suppose that $t \in S$.
Consider the thresholded region $R_{t}$, defined by
\begin{equation}
  \label{eq:def-region-threshold}
  R_{t}
  =
  \lrset
  {
    p \in R
    \mid
    I\lr{p} < t
  }
  ,
\end{equation}
which is itself an extremal region, and for which we have that
\begin{equation}
  \label{eq:def-region-threshold}
  \max_{p \in R_{t}} I\lr{p}
  <
  t
  .
\end{equation}
We note that $R_{t} \subseteq R$ for all $t \in S$.
We also note that when $t_{1} \leq t_{2}$, we have that
$R_{t_{1}} \subseteq R_{t_{2}}$; that is, the thresholded regions form an
increasing (by set inclusion) sequence of subsets of $R$.
For any increasing chain $t_{1} < \dotsc < t_{n}$ in $S$, we have
\begin{equation}
  \label{eq:def-region-threshold-sequence}
  \emptyset
  \subseteq
  R_{t_{1}}
  \subseteq
  \dotsb
  \subseteq
  R_{t_{n}}
  \subseteq
  R
  .
\end{equation}
In the MSER approach, the stability of an extremal region $R$ is measured by
examining the change in the cardinality of $R_{t}$ with the change in the
threshold $t$.
That is, for a particular threshold $t \in S$ and threshold step $\delta \in S$,
such that $t - \delta, t + \delta \in S$, the rate of growth of the extremal
region $R$ is given by
\begin{equation}
  \label{eq:def-region-growth}
  G_{\delta}\lr{R; t}
  =
  \frac
  {
    \lrabs
    {
      R_{t + \delta}
      \setminus
      R_{t - \delta}
    }
  }
  {
    \lrabs{R_{t}}
  }
  .
\end{equation}
An extremal region $R_{t_{0}}$ is then said to be maximally stable if
$G_{\delta}\lr{R; t}$ has a local minimum at $t = t_{0}$.

\subsection{Task 1}
\label{sec:imp-task-1}

\subsection{Task 2}
\label{sec:imp-task-2}

\section{Validation Performance}
\label{sec:val-per}

\subsection{Task 1}
\label{sec:val-task-1}

\subsection{Task 2}
\label{sec:val-task-2}

\section{Conclusion}
\label{sec:conclusion}

\clearpage

\appendix

\section{Source Code}
\label{sec:source-code}

\end{document}
